%%%%%
%%%%% by James Long
%%%%%
%%%%% brief report on noisification with ASAS and
%%%%% plans for simulations + ASAS experiments
%%%%%


\documentclass[10pt]{article}
\usepackage{verbatim, amsmath,amssymb,amsthm,graphicx}
\usepackage[margin=.5in,nohead,nofoot]{geometry}
\usepackage{sectsty}
\usepackage{float}
\sectionfont{\normalsize}
\subsectionfont{\small}

\title{}
\date{}
\author{}
\newtheorem{theorem}{Theorem}[section]
\newtheorem{definition}{Definition}[section]
\newtheorem{example}{Example}[section]

\newcommand{\argmin}[1]{\underset{#1}{\operatorname{argmin}}\text{ }}
\newcommand{\argmax}[1]{\underset{#1}{\operatorname{argmax}}}
\newcommand{\minimax}[2]{\argmin{#1}\underset{#2}{\operatorname{max}}}
\newcommand{\bb}{\textbf{b}}

\newcommand{\Var}{\text{Var }}
\newcommand{\Cov}{\text{Cov }}


\newenvironment{my_enumerate}{
\begin{enumerate}
  \setlength{\itemsep}{1pt}
  \setlength{\parskip}{0pt}
  \setlength{\parsep}{0pt}}{\end{enumerate}
}



% Alter some LaTeX defaults for better treatment of figures:
    % See p.105 of ``TeX Unbound'' for suggested values.
    % See pp. 199-200 of Lamport's ``LaTeX'' book for details.
    %   General parameters, for ALL pages:
\renewcommand{\topfraction}{0.9}% max fraction of floats at top
\renewcommand{\bottomfraction}{0.8}% max fraction of floats at bottom
    %   Parameters for TEXT pages (not float pages):
    \setcounter{topnumber}{2}
    \setcounter{bottomnumber}{2}
    \setcounter{totalnumber}{4}     % 2 may work better
    \setcounter{dbltopnumber}{2}    % for 2-column pages
    \renewcommand{\dbltopfraction}{0.9}% fit big float above 2-col. text
    \renewcommand{\textfraction}{0.07}% allow minimal text w. figs
    %   Parameters for FLOAT pages (not text pages):
    \renewcommand{\floatpagefraction}{0.7}% require fuller float pages
    % N.B.: floatpagefraction MUST be less than topfraction !!
    \renewcommand{\dblfloatpagefraction}{0.7}% require fuller float pages

    % remember to use [htp] or [htpb] for placement





\begin{document}
\section{Noisification with ASAS Data}

%%%%% to include
%% 1. 4 error rates
%% 2. 2 (or 4) confusion matricies
%% 3. variable importance plots

I analyzed 13,735 curves belonging to 8 classes in the ASAS data set. I divided these curves into training and test sets of equal size. I noisified all curves in both sets by selecting only the first 35 flux measurements. I built a clean classifier (Random Forest) on the features derived from the well sampled curves in the training set and a noisified classifier on the features derived from 35 epoch curves in the training set. Then I applied both classifiers to both test sets to come up with 4 error rates and 4 confusion matrices. 

\subsection{Error Rates}

The largest class is Close Eclipsing Binary Systems which make up about 21\% of the data so assuming uninformative features, the Bayes error rate is about .79.

% latex table generated in R 2.11.1 by xtable 1.5-6 package
% Wed Jan 26 01:10:21 2011
\begin{table}[ht]
\begin{center}
\begin{tabular}{rr}
  \hline
 & Error Rate \\ 
  \hline
Clean Train / Clean Test & 0.204 \\ 
  Clean Train / Noisy Test & 0.739 \\ 
  Noisy Train / Noisy Test & 0.500 \\ 
  Noisy Train / Clean Test & 0.788 \\ 
   \hline
\end{tabular}
\caption{Error rates for clean / noisy training data used to predict clean / noisy test data}
\end{center}
\end{table}



\subsection{Confusion Matrices}

True class are columns, predicted class are row. These confusion matrices are normalized by column so the (i,j) entry represents the fraction of sources belonging to class j that were classified as class i.

\begin{figure}[H]
\begin{center}
\includegraphics[height=5in,width=4in]{asas_confusion_clean_on_clean.pdf}
\caption{Clean training set, clean test set.\label{fig:asas_confusion_clean_on_clean}}
\end{center}
\end{figure}

\begin{figure}[H]
\begin{center}
\includegraphics[height=5in,width=4in]{asas_confusion_clean_on_noisy.pdf}
\caption{Clean training set, noisy test set.\label{fig:asas_confusion_clean_on_noisy}}
\end{center}
\end{figure}

\begin{figure}[H]
\begin{center}
\includegraphics[height=5in,width=4in]{asas_confusion_noisy_on_noisy.pdf}
\caption{Noisy training set, noisy test set.\label{fig:asas_confusion_noisy_on_noisy}}
\end{center}
\end{figure}

\begin{figure}[H]
\begin{center}
\includegraphics[height=5in,width=4in]{asas_confusion_noisy_on_clean.pdf}
\caption{Clean training set, noisy test set.\label{fig:asas_confusion_noisy_on_clean}}
\end{center}
\end{figure}



\subsection{Variable Importance Measure}

The classifiers are using very different sets of features. The noisified classifier uses primarily non L-S features.

\begin{figure}[H]
\begin{center}
\includegraphics[height=5in,width=4in]{var_imp_clean.pdf}
\caption{Feature importances for the clean classifier.\label{fig:var_imp_clean}}
\end{center}
\end{figure}

\begin{figure}[H]
\begin{center}
\includegraphics[height=5in,width=4in]{var_imp_noisy.pdf}
\caption{Feature importances for the clean classifier.\label{fig:var_imp_noisy}}
\end{center}
\end{figure}

\end{document}



